\documentclass{article}
\usepackage{amsmath}
\usepackage{amssymb}
\usepackage{amsthm}
\usepackage{amsfonts}
\title{Problem 585}
\author{Curtis Bradley}
\date{March 2024}

\begin{document}

Let $s$ be the length of each side of the square $ABCD$. Place the square in $\mathbb{R}^2$ as follows: $A(0,s)$, $B(0,0)$, $C(s,0)$, and $D(s,s)$. With the pole at the point $(x,y)$ such that $AP$ = 13ft, $BP$ = 5 ft, and $CP$ = 17 ft. We can construct a circle of radius 13 ft centered at $A$, a circle of radius 5 ft centered at $B$, and a circle of radius 17 ft centered at $C$, such that all three circles intersect at point $P$. The equations of these three circles are:
\begin{equation}
    x^2 + (y-s)^2 = 13^2
\end{equation}
\begin{equation}
    x^2+y^2 = 5^2
\end{equation}
\begin{equation}
    (x-s)^2 + y^2 = 17 ^2
\end{equation}
By substituting (2) into (1) and (3), we find 

\begin{center}
    $x = \frac{s^2-264}{2s}$ and $y=\frac{s^2 - 144}{2s}$
\end{center}
Substituting both into equation (2) gives us
$s^4 - 458s^2+ 45216 = 0$. By quadratic formula $s^2 = 144$ or $s^2 = 314$. Because city ordinance requires a square field of at least 300 square feet only the second option is considered. Using the x and y equations, we find $P = (\frac{25}{\sqrt{314}},\frac{85}{\sqrt{314}})$. By graphing a circle of radius 21 feet centered at $P$ it can be seen that the circle encompasses the area of the entire square. Through calculus optimization it can found that the closest the circle gets to the square is a distance of $.191$ ft. Therefore Fido has a total roaming area of 314 square feet satisfying the ordinance.


\end{document}

